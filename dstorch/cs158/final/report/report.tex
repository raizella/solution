\documentclass[11pt]{article}

\title{Building a Search Engine: Final Report}
\author{David Storch \\ login: dstorch}
\date{6 May 2011}

\usepackage{fullpage}
\usepackage{amsmath}
\usepackage{amssymb}

\parskip 7pt
\parindent 0pt

\begin{document}

\maketitle

\section{Modified preprocessing}

\subsection*{Zones}

The indexing program, \texttt{createIndex.sh}, now offers an implementation of zones.
Namely, it treats the title, the top section of the page (i.e. the \textit{header section} before any subsections begin), and the remainder
of the page text as three different zones. There are separate indexes for each of these zones, allowing
\texttt{queryIndex.sh} to weight tf-idf scores higher in the title and top section than in the remainder
of the page. It is useful to give high weight to the header section because this section of a wiki article generally contains core terminology.
I implemented zones by separating the header from the remainder of the
text, and then indexing each part separately.

My implementation of zones does not incur additional time overhead because
the total number of terms in the collection that need to be processed is unchanged. This was confirmed
in practice: producing zone indices did not increase the processing time for \texttt{createIndex.sh}.

There is a small space overhead because we need to store a separate dictionary for each index, and these dictionaries
share a large number of terms. Furthermore, if a word occurs in both the header and the body of a page, we must store its
docID in both indices. In the worst case, the dictionary term redundancy is $O(t)$,
where $t$ is the number of terms in the collection, and the docID redundancy is $O(N)$ where $N$ is the number of documents
in the collection. Where $l$ is the number of tokens in the entire collection, $N < t << l$, meaning that the space cost
is small in comparison to the size of the full collection. The index from part 2 was 253,700,823 bytes whereas the
combined indices using a zone implementation was 259,472,965 bytes---only 2\% larger.

\subsection*{Stubs}

\texttt{createIndex.sh} identifies docIDs corresponding to very short articles that are marked
as stubs. These identifications are passed to \texttt{queryIndex.sh}, which decreases the score of
stub articles. My search engine from Part 2 would sometimes favor very short documents due to document
length normalization---if search terms appear in a document that is only 50 words long, the search terms
comprise an unnaturally large proportion of the terms in the document. Furthermore, very short articles are often
not the most relevant or authoritative. There is no additional time cost for underweighting stubs, and the space cost is $O(N)$, which as
we have already discussed is not significant relative to the size of the full collection.

\subsection*{Wiki Markup}

The wiki markups are parsed by \texttt{createIndex.sh} so as to remove semantically useless
tags. For instance, tags such as ``$<$div class=`references-small'/$>$'' are removed, and will not be
indexed. However, tags that may have useful semantic content---such as references to related webpages
or written documents---\textit{are} indexed.

This was implemented by preprocessing the text with regular expressions prior to indexing. Therefore,
this requires an additional scan through the entire collection---but does not change the overall
time complexity. Removing unneeded wiki tags saves space proportional to the number of excised tags.

\subsection*{Stop Words}

In order to improve phrase queries that involve stop words, I decided to produce a shorter stop
words list. My new list consists of all terms in the lexicon that appear at least once in 75\% of the
documents in the collection. The resulting list of 15 words appears in the file \texttt{stopWords.out}. A term that appears
in more than three-quarters of the documents is extremely unlikely to contribute to rankings in a meaningul
way. Using the inverted stop words lists increased the size of the inverted index by 13\% (from 259,472,965
bytes to 293,744,695 bytes).

I compared rankings on my standard query set using the old and new stop words, and found that the rankings
 are identical for most queries. Therefore, removing stop words does not have a major effect on ranking results.
However, there is a significant imrpovement for select queries because stop words in the original
list, such as ``image'' and ``c'', have important meaning in the context of computer science. With my new 
stop word set, the first two results for the query \texttt{c shell} are the articles ``C Shell'' and ``C (programming language)'',
whereas with the original stop words these most relevant articles are ranked lower.
The smaller stop words set thus supports better queries, at the expense of a larger
inverted index.

\section{Ranking algorithm for \texttt{queryIndex}}

The core ranking algorithm is a weighted average of header section tf-idf, page body tf-idf,
PageRank, and Multinomial Naive Bayes classification. The algorithm is described step-by-step below:

\subsection*{Weighted Average}

First, the set of pages $\mathcal{P}$ matching the query is retrieved, according to the query type.
Next the max tf-idf for the header zone and the max tf-idf for the page zone are computed over
the set of search hits $\mathcal{P}$. The zone scores for each document $d \in \mathcal{P}$ are normalized
by the respective maxes, giving the header score h$(d)$ and the body score b$(d)$. This step ensures that
the both h$(d)$ and b$(d)$ are ``on even footing'', with their relative contributions to the overall
ranking controlled only by the weights in our weighted average.

Unlike the tf-idf scores, the PageRank scores spanned a range of about three orders of magnitude,
from about $10^{-3}$ to $10^{-6}$. In order to compress this range, I take the page rank as

\[\text{pr}(d) = \frac{-1}{\log(\text{pr}'(d))}\]

where $\text{pr}'(d)$ denotes the original PageRank of document $d$. These new PageRank scores
are then normalized by the highest such score in the result set $\mathcal{P}$.

Instead of classifying the query, I use the multinomial naive Bayes classifications
of the \textit{search results} to contribute to the weighting scheme.\footnote{We found in the that our implementation of the
Multinomial Naive Bayes classifier could classify unseen documents with an error rate of about 40\%}
After retrieving the query result set $\mathcal{P}$,
the fraction of documents classified within each category is determined.
The score $\text{mnb}(d)$ for a document $d \in \mathcal{P}$ is 

\[\text{mnb}(d) = \frac{C_{d}}{|\mathcal{P}|}\]

where $C_{d}$ denotes the number of documents in $\mathcal{P}$ that are classified in the same category as $d$.
Using this scheme, the MNB term does not have much of an effect on the overall ranking unless a large proportion
of the documents in $\mathcal{P}$ have the same classification. In this case, the documents that do not belong to
the most common class are at a relative disadvantage, and should be sifted down to a lower position in the ranking.

The ranking was computed as a weighted average

\[\text{rank} = w_{1} \cdot \text{h}(d) + w_{2} \cdot \text{b}(d) + w_{3} \cdot \text{pr}(d) + w_{4} \cdot \text{mnb}(d)\]

where $0 \leq \text{rank} \leq 1$ and $w_{1} + w_{2} + w_{3} + w_{4} = 1$.

\subsection*{Finding the Weights}

The weights are initialized as follows: $w_{1} = 0.45$, $w_{2} = 0.4$, $w_{3} = 0.1$, $w_{4} = 0.05$. I determined these
default weightings by experimenting as described in Question 3 below.

The weights are updated for each document depending on specific document characteristics. The body tf-idf weight $w_{2}$ is decreased
when the article is tagged as a stub. When the header section is very short (less than 1000 characters), the weight given to the header
is decreased, avoiding a bias in favor of articles with short headers.

Most importantly, the PageRank weight is determined by the average inverse document frequency of the query terms. Since query specificity is
associated with a high idf score, the PageRank weight $w_{3}$ is defined by a function that increases linearly with \textit{decreasing} idf. In order to compensate, $w_{1}$ and
$w_{2}$ increase linearly with \textit{increasing} idf.

\subsection*{Additional Weighting}

Finally, if there are $k$ terms in the query that match terms in the document title, the overall score is multiplied by $1.3^{k}$.
This strongly favors documents whose titles share words with the query.

\section{Experimental set-up for evaluating search relevance}

In addition to printing the top 10 docIDs selected by the search engine, I altered
my search engine to print the final score, the contributing terms that led to this score,
and the title of the document. This allowed me to adjust the default weights $w_{1}$, $w_{2}$,
$w_{3}$, and $w_{4}$, as well as to observe some trends in the rankings. For example, I noticed
that on some general queries, unrelated articles with low tf-idf scores could make it into
the top ten results due to the boosted PageRank weight. That is, a non-relevant document
containing a single instance of a query term could still be ranked highly due to PageRank.
This led me to establish the heuristic that if the
weighted PageRank score is 10 times larger than the weighted idf-score, then the PageRank is
decreased by 50\%---an effective technique for eliminating non-relevant results with inflated
PageRank.

I also ran repeatedly ran my search engine on a standard set of queries.\footnote{The queries appear in
the file \texttt{queries.dat}. They are a combination of the provided queries, and new
queries that I have generated.} I then ran the script \texttt{evaluateTestQueries.py} on the results in order
to compare the results to my search engine from Part 2. This was useful both as a sanity check to
make sure that my weighted ranking scheme was producing reasonable results, and to evaluate my progress as
I made changes to the algorithm.

%\begin{math}
%<\text{docID}> <\text{final score}> w_{1}\text{h}(d) w_{2}\text{b}(d) w_{3}\text{pr}(d) <\text{title}>
%\end{math}
%
%I also generated a long list of standard test queries which I could run
%
%I compiled a short list of queries by which the ranking scheme could
%be manually inspected and analyzed for accuracy of results. These queries
%are summarized in the table below.
%
%
%\begin{tabular}{|l|l|}
%\hline
%\textbf{query} & \textbf{description} \\
%\hline
%\texttt{computer science} & general term with lots of hits; \\
%			  & many relevant hits so only the most \\
%			  & general or the most authoritative should \\
%			  & be chosen \\ \hline
%\texttt{microsoft windows} & a specific but popular query which should \\
%			  & have lots of hits; the best ranking selects \\
%			  & extremely relevant articles first such as \\
%			  & ``Microsoft Windows'' and ``Windows XP'' \\ \hline
%\texttt{liverpool blue coat school}
%			  & an obscure query which should have the single \\
%			  & relevant article ranked first \\ \hline
%\texttt{PageRank AND Google}
%			  & a highly specific query which should return \\
%			  & authoritative articles such as ``Google'', \\
%			  & ``PageRank'', and ``History of Google'' \\ \hline
%\texttt{the best apple products to buy}
%		 	  & a good search engine will effectively ignore the \\
%			  & unimportant words in this query---it will return \\
%			  & results related to apple products \\ \hline
%\texttt{hello kitty}
%			  & a query outside the scope of the collection; \\
%			  & a good search engine will be able to locate \\
%			  & the few references to this topic inside the \\
%			  & collection \\ \hline
%\texttt{``image id''}
%			  & these terms are very common due to wiki markup tags, \\
%			  & and appear in the original stop words list \\
%\end{tabular}
%
%These queries appear in the file \texttt{myQueries.dat}, and can be piped into
%the search engine in order to evaluate its performance. In addition to these
%selected queries, I ran my results for all provided queries for part 2 against
%the expected queries in order to make sure that there was consistently some
%overlap. This was a sanity check to make sure that my weighted
%ranking scheme was producing reasonable results---see Question 5 for more
%details.

\section{Analysis of final ranking scheme}


\subsection*{One word queries}

One word queries perform well for both obscure and general terms. Both the
general term \texttt{computer} and the specific term \texttt{BBN} (referring to the Cambridge, Massachusetts
technology company) give reasonable results. It is difficult to find a one-word query that
does not succeed. Even extremely vague queries like just the letter \texttt{N} produce some reasonable
results (``N-gram'' and ``N-jet'').

\subsection*{Free text queries}

The search engine performs very well when given ``popular queries''---queries related
to a topic which appears frequently in the collection and which will be best described
by a few detailed articles. For example,the query \texttt{Microsoft Windows} returns
``Microsoft Windows'', ``Windows 2000'', and ``Windows Vista'' as the three top hits.

The engine also does extremely well for obscure queries targeted at particular documents.
For instance, \texttt{ricoh caplio gx100} returns ``Ricoh Caplio GX100'' followed by
several other wiki pages on Ricoh cameras.

The performance is not as good, however, for queries such as \texttt{important
algorithms to learn} which have words which contribute very little to the meaning of the query.
The top results for this query are ``Meta learning (computer science)'' and ``Student approaches
to learning''. The search engine does not do any semantic analysis to determine that the querier
actually wants to find documents relevant to the term ``algorithms'' and not to
the term ``learn''.

\subsection*{Phrase queries}

Most phrase queries work smoothly, especially given the reduced stop words list. For instance,
the general query \texttt{``computer science''} and the more specific phrase \texttt{``most common configuration''}
both produce good results. Whereas \texttt{``computer science''} identifies the principle articles
about the discipline of computer science, \texttt{``most common configuration''} identifies the handful of documents
which contain the query phrase and ranks them reasonably.

Phrase queries can perform poorly if the query terms have imbalanced tf-idf rankings. For example, the
results for the query \texttt{``new algorithm''} are influenced heavily by the term ``algorithm'' but have
nothing to do with the query term ``new''. 

\subsection*{Boolean queries}

Boolean queries turn out swimmingly if the invidual terms in the query are closely related. For example,
\texttt{google AND pagerank} turns up ``PageRank'', ``Google'', ``List of Google Products'', and
``History of Google'' as the first four hits---clearly the most relevant responses to the query.

If terms are somewhat randomly conjoined by boolean operators, the results are noticably worse.
Consider the query \texttt{tree OR situation AND bill}. The ranking promotes articles like
``Phylogenetic tree'' and ``Veteran tree'', but there is no obvious relationship in the
top articles to the terms ``situation'' and ``bill''.

\subsection*{Wildcard queries}

For the query \texttt{t*pedo}, all of the search results are indeed torpedo-related.

One-word queries that refer to a variety of subjects also succeed. For example, the search
\texttt{bio*} yelds ``Biostar'', ``Biotope'', ``Biome'', and ``Bioregionalism''. In this case the success
is due to high tf-idf weights---the prefix ``bio'' is very common in these articles. If we allow PageRank
to take over for an overly general query, however, the results can become seemingly random. This is true
for the query \texttt{tope*} because the character string ``tope'' can appear in diverse contexts. The hits
for \texttt{tope*} involve both Topeka, Kansas and j-topes (j-dimension polytopes from mathematics).

\subsection*{Wildcard phrase queries}

Wildcard queries perform well when the asterisk can be matched to a relatively small number
of terms. For instance, \texttt{``tur* machine''} returns ``Turing machine'', ``Non-deterministic Turing machine'',
and ``Multi-track Turing machine'' as its three top hits.

The results can be bad when the wildcard has many possible matches. The top hits for \texttt{``text class*''}
are ``Polymorphic association'' and ``Member variable''. Such bad results can occur because the weighting scheme will
 treat the query as general, giving additional weight to PageRank.
The query is intended to match ``text classification'', a specific topic which will have a low PageRank compared
to more general topics in computer science which use variants of the term ``class''.
Furthermore, the wildcard term may influence the results more than the other terms in the query. 


\section{Comparison to the rankings from Part 2}

For my test query set, the average number of shared results for a given query (ignoring order) is
5.98, so there is a fair amount of similarity between the two search engines. However, the examples
below demonstrate that the change represents an improvement from the original search engine implementation.
There is noteworthy improvement for the following types of queries:

\begin{itemize}
\item Phrase queries or free text queries involving stop words.
\item General queries whose terms appear widely throughout the collection.
\item Queries matching the title of an article.
\item Queries whose terms appear frequently in the header section of an article.
\end{itemize}


\subsection*{One word queries}

The query \texttt{windows} clearly demonstrates the benefits of the advanced search engine implementation.
The naive search engine returns ``Window'' (the architecture kind) as its top hit. The advanced implementation, on
the other hand, is able to infer that the top hit should be ``Microsoft Windows''. This is due primarily to
the extremely high PageRank of the ``Microsoft Windows'' page. However, the repitition of the term ``windows''
in the header section of the article also leads to a high zone score for the header section.

The original implementation yields ``David Peleg (computer scientist)'' as the top-ranked document in response to the query
\texttt{dijkstra}. David Peleg won an award named after Dijkstra, and therefore the article has little to do with
Edsger W. Dijkstra himself. On the other hand, the improved search
engine ranks ``Structured programming'' first, a concept pioneered by Dikjstra. These improved results
largely arise from zones and identification of stubs. The ``David Peleg'' article is very short,
whereas ``Structured programming'' is longer and mentions Edsger Dijkstra several times in the header
section. In this way, the query \texttt{dijkstra} demonstrates the affects of zones and stubs preprocessing.

\subsection*{Free text queries}

The query \texttt{b-tree} demonstrates the utility of shrinking the stop words set. The original implementation
comes up with ``List of data structures'' and general articles about trees. The advanced implementation registers
the ``b'' as a term rather than a stop word, and returns ``B-tree'' as the top hit, with ``B sharp tree'' and ``B*-tree'' close behind.

The original search engine returned lots of bad results for the query \texttt{computer science}. The hits
``E-science'' and ``E-social science'' are probably not what the end-user would want to find. This pollution of search
results stems from the fact that virtually the entire collection contains references to computer science. A simple
tf-idf engine will therefore just return the document with the most frequent usage of the terms ``computer'' and ``science''---not a
very useful measure of relevance for the end-user. The advanced search engine, on the other hand, returns ``Computer science'',
``Bachelor of computer science'', and ``Computational science'' as the top hits. PageRank has used link analysis to
determine the value of an article $a$ in terms of the articles which link to $a$. This provides
a measure of content value that goes beyond whether the text contains the query terms.

The advanced search engine similarly improves on most general queries, especially about central topics in
computer science. We can observe a similar improvement for the query \texttt{mac computer}.

\subsection*{Phrase queries}

Running the phrase query \texttt{``structured query language''} on both search engines provides
a good example of the advantages of the new implementation. The old search engine highly ranks
``Microsoft Query by Example'' and ``Chess Query Language''. The new search
engine implementation ranks ``Structured query language interface'', ``.QL'', and ``SQL'' as the top three hits.
``Microsoft Query by Example'' is a stub article, resulting in a lower ranking by the advanced
search engine.

For the query \texttt{``music computer''}, the original implementation comes up with generic computer-related
results such as ``Generative grammar'' and ``List of computer magazines''. In constrast, the advanced search engine
is able to find results that are related to both computers and music---``Orion (music software)'', for example.
The contributions from the various terms in the weighted average suggest that this was an effect of both
PageRank and multinomial naive bayes classification.

\texttt{``Central processing unit''} produces great results on the advanced implementation; the wiki articles
``Central processing unit'', ``Process management (computing)'', and ``Arithmetic logic unit'' are three of the
top four hits. The naive search engine produces ``quantum processing unit'' as its best result along with
several other less relevant results. 

\subsection*{Boolean queries}

For the original search engine, the query \texttt{united AND parliament} does not produce many relevant
results beyond unfamiliar names of statesmen. On the other hand, the improved implementation yields
pages including ``United Kingdom'' and ``House of Lords''.

The wiki article ``IBM System z'' contains a detailed introductory section which repeatedly uses
the terms IBM and zSeries. We expect to find that the ranking of this article will be improved
by the advanced implementation for the query \texttt{IBM AND zSeries}. Indeed, we find that
``IBM System z'' is the first hit.

When passed the query \texttt{javascript OR web AND design}, the naive search engine produces
reasonable results. However, almost all of the results are related to Java script and not to
web design. The advanced implementation does a better job of including results relevant to
both sides of the OR statement. The results include ``Unobtrusive Javascript'' and ``Javascript engine'',
but also ``Tableless web design'' and ``Web development''.

\subsection*{Wildcard queries}

Having parsed out the <ref> tags in the wiki markup, the query \texttt{ref*} turns up much more useful
results (``reflection (computer science)'', ``reference (computer science)'') than the naive implementation.

The new implementation also does better for \texttt{vapor*}, turning up results such as ``Reboiler'' rather
than ``Prognostic variable'' and ``Duke Nukem Forever''. Examining the relative contributions from the
various terms in the weighted average shows that the ranking was larger determined by the text in the header
section of the pages.

The \texttt{tur* machine} example from the previous problem, when passed to the naive search engine, does not output
any Turing machine-related results. In contrast, the top results for
the advanced implementation are articles about Turing machines.


\end{document}
