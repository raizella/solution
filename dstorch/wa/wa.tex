\documentclass[10pt]{article}
\usepackage{url}
\usepackage{graphicx}
\usepackage{verbatim}
\usepackage{fancyvrb}
\usepackage{amsmath, amssymb}
\setlength\textwidth{6.5in}
\setlength\oddsidemargin{0in}
\setlength\evensidemargin{0in}
\pagestyle{empty}

\newcommand{\olya}[1]{\textbf{Olya: #1}}
\newcommand{\eli}[1]{Eli: \textbf{#1}}
\newcommand{\alb}[1]{\textbf{Alb: #1}}
\newcommand{\ignore}[1]{}

\newcommand{\file}[1]{\texttt{#1}}
\newcommand{\code}[1]{\texttt{#1}}

\fvset{numbers=left,frame=single}

\begin{document}

\title{\vspace{-15mm}{\small Brown~University --- CSCI~1580 --- Spring~2013}\\Warm-Up Assignment}
\author{Due: 11:59am, 01 Feb 2013}
\date{}
\maketitle
\pagestyle{plain}
\thispagestyle{empty}

\section*{Overview}

% This warm-up assignment has the following goals:
% \begin{itemize}
% %     \item you will get acquainted with Python,
%     \item you will test your ability to write
%         specification-compliant code, and
%     \item you will understand how the submission script works.
% \end{itemize}
% You will need all the above skills
% to successfully develop the Course Project
% ``Building a Web Search Engine''!

This warm-up assignment has two goals:
you will test your ability to write
specification-compliant code,
and you will understand how the submission script works.
You will need both of the above skills
to successfully develop the Course Project
``Building a Web Search Engine''!

In this assignment you will write a toy program
to solve anagrams.
You are given a dictionary (a set of words),
and you should write a program that tells you
if the sequence of characters you pass in
is the anagram of some word in the dictionary.
In this case, you should output the matching word(s).
For example, if you query for ``cta'',
you might get ``cat'' and ``act''.
A detailed specification will be given
in Section~\ref{spec}.

As explained in Section~\ref{submission},
you will submit your files
using the same procedure that
you will use later for the Course Project,
so please follow it strictly.

\emph{You should submit this assignment
in order to be registered
in the evaluation system of the Course.}

Please read carefully the whole document
before starting coding.
If you are in doubt, ask the TAs.

\section{Specification}
\label{spec}

\subsection{General Requirements}
\label{general}

As mentioned in the overview,
you are given a set of words,
the ``dictionary'',
and you should write a program that
returns the word(s) in the dictionary
that are anagrams of the word(s) you pass in.
To solve the task, you will use the following strategy.

First, you will write a program,
% \file{createIndex.py},
\file{createIndex},
that reads each word from the dictionary,
and returns a sorted index of entries,
with each entry corresponding
to exactly one word of the dictionary.
Each entry should be a pair ``\code{asciisort(word)$|$word}'':
the part after the ``\code{$|$}'' is the dictionary word,
while the part before the ``\code{$|$}'' is
the concatenation of its characters,
sorted in increasing ASCII code order.
The whole index must be sorted,
first according to \code{asciisort(word)}
and, in case of a tie, according to \code{word}.

Then, you will write a second program,
% \file{queryIndex.py},
\file{queryIndex},
that finds out whether there exists
one or more words in the dictionary
that are the anagrams of a given query,
using the index built by
% \file{createIndex.py}.
\file{createIndex}.

\subsection{Programming Language}
\label{language}

You can use any programming language you like,
provided that its compiler and interpreter
are available on any CS machine.
\emph{However, we strongly suggest you to code in Python,
as it significantly reduces the programming overhead.}

To run your programs,
we will execute two Bash scripts,
\file{createIndex.sh} and \file{queryIndex.sh},
which you should modify by
adding the proper command line for invoking your programs.
For example, if you choose to code in Python,
and write your main program as \file{queryIndex.py},
your \file{queryIndex.sh} file might look as follows:
\begin{figure}[ht]
\begin{Verbatim}
#!/bin/bash
python queryIndex.py $@
\end{Verbatim}
% \caption{Example of the Bash script \file{queryIndex.sh}}
\label{fig:queryIndex}
\end{figure}


\subsection{Input/Output Specification}
\label{input}

You can presume that the dictionary contains words,
one per line, where ``word'' is a non-empty string on the alphabet \code{[0-9A-Za-z ]}.
(Note that a word might contain spaces.)
The same assumption holds for the queries too.
For example,
``\code{orwell}'',
``\code{Orwell}'',
``\code{ORWELL}'',
``\code{1984}'',
``\code{\ \ }'' (two spaces),
``\code{Animal Farm}'',
are all legal and distinct words.
You do not need to worry
about checking for valid/invalid/duplicated words in your code.

The dictionary is a sequence of words,
separated by a newline character ``\code{\textbackslash n}''.
% Your \file{createIndex.py} program should read
Your \file{createIndex} program should read
the dictionary from the standard input
and output the generated index on the standard output,
thus allowing file redirection.
Note that the number of lines of \file{myIndex.txt}
should be the same as of \file{myDictionary.txt}.
See below for a complete example.

You will pass the name of the dictionary file
as the first argument on the command line
% invoking your second program, \file{queryIndex.py}.
invoking your second program, \file{queryIndex}.
This program will load the index
and then accept queries from the standard input,
each separated by a newline character,
until the input is closed (e.g., by ``CTRL+D'' or EOF).
If the given query is the anagram of a word in the dictionary,
then your program should write on the standard output
the list of all matching words, separated by ``\code{$|$}''.
If no match is found, it should output a blank line.
As before, the number of lines of \file{myAnagrams.txt}
should be the same as of \file{myQueries.txt}.

\subsection{Example}
Consider the dictionary file \file{myDictionary.txt} in Figure~\ref{fig:myDictionary}.
Invoking:
% \begin{verbatim}
%      $ python createIndex.py < myDictionary.txt > myIndex.txt
% \end{verbatim}
\begin{verbatim}
     $ ./createIndex < myDictionary.txt > myIndex.txt
\end{verbatim}
should produce the index file \file{myIndex.txt}
as in Figure~\ref{fig:myIndex}
(note the leading spaces):
\begin{figure}[ht]
    \begin{minipage}[b]{0.45\linewidth}
        \VerbatimInput{dic.txt}
        \caption{\file{myDictionary.txt}}
        \label{fig:myDictionary}
    \end{minipage}
\hspace{1cm}
    \begin{minipage}[b]{0.45\linewidth}
        \VerbatimInput{ind.txt}
        \caption{\file{myIndex.txt}}
        \label{fig:myIndex}
    \end{minipage}
\end{figure}

If we submit the sample queries of
file \file{myQueries.txt} as in Figure~\ref{fig:myQueries},
by invoking:
% \begin{verbatim}
%     $ python queryIndex.py myIndex.txt < myQueries.txt > myAnagrams.txt
% \end{verbatim}
\begin{verbatim}
    $ ./queryIndex myIndex.txt < myQueries.txt > myAnagrams.txt
\end{verbatim}
we should get the anagrams file \file{myAnagrams.txt} listed
in Figure~\ref{fig:myAnagrams}.
\begin{figure}[ht]
    \begin{minipage}[b]{0.45\linewidth}
        \VerbatimInput{que.txt}
        \caption{\file{myQueries.txt}}
        \label{fig:myQueries}
    \end{minipage}
\hspace{1cm}
    \begin{minipage}[b]{0.45\linewidth}
        \VerbatimInput{ana.txt}
        \caption{\file{myAnagrams.txt}}
        \label{fig:myAnagrams}
    \end{minipage}
\end{figure}

% \newpage
\section{Data}
\label{data}

For this assignment you will need the following files located in
\url{/course/cs158/data/warmup/} :
\begin{itemize}
	\item \url{smallDictionary.txt} : a small dictionary for quick testing;
	\item \url{smallQueries.txt} : sample queries for the small dictionary;
	\item \url{smallAnagrams.txt} : expected result for the above queries;
	\item \url{largeDictionary.txt} : a full dictionary;
	\item \url{largeQueries.txt} : sample queries for the large dictionary;
	\item \url{largeAnagrams.txt} : expected result for the above queries;
	\item \url{createIndex.sh} : sample Bash script to invoke your \file{createIndex} program;
	\item \url{queryIndex.sh} : sample Bash script to invoke your \file{queryIndex} program;
	\item \url{readme.txt} : a plain text file you should fill in with your information.
\end{itemize}


\section{Collaboration Policy}
\label{collab}

% You can work on this warm-up assignment and
% on the Course Project in pairs.
% You should work on all the programming assignments
% with the same team-mate,
% and only one submission per team is required,
% unless explicitly noted.
% \emph{We strongly suggest you to consider working with another student.}

% Please check the web page of the Course for links to online Python resources.

You should work on this warm-up assignment individually.
Please use \url{http://groups.google.com/group/brown-cs158-spring2011} or
email \texttt{cs158tas at cs dot brown dot edu} if you have any questions.


\section{Submission}
\label{submission}
Please copy all the files mentioned below
into a separate directory, \code{cd} there, and
run the handin script from that directory.
Be aware that since the handin script copies recursively
all the files from the directory it is executed from,
if you run the script from, say, your home directory,
it will handin all your files and directories!

Also, you can name any files you do not want to include
as \file{.dat} or \file{.out} files. These files will be excluded
from your handin, so make sure that nothing that your handin
depends on uses these files.

The directory from where you run the handin script should contain only:
\begin{itemize}
% 	\item Any Python source code (\file{.py}) you wrote.
	\item Any source code you wrote.
	\item The two Bash scripts \file{createIndex.sh} and \file{queryIndex.sh},
		modified to call your programs.
	\item The plain text file \file{readme.txt}, filled in
        with the required information.
\end{itemize}

Please submit the above files using the following command:
\begin{verbatim}
    $ /course/cs158/bin/cs158-handin warmup
\end{verbatim}


\section{Evaluation}
\label{evaluation}

We will run your programs 
on a CS machine, using
\file{largeDictionary.txt} and
\file{largeQueries.txt}.
We will verify that your programs produce
the \emph{exact} expected output,
by running the command line
utility \file{diff}.
Therefore, please be sure to submit
properly working code.
In particular, check for unwanted prompts,
debugging messages, newlines
that you might have forgotten in your code.
\emph{It is your responsibility to make sure that
your code runs on the department machines,
satisfies naming and usage criteria described in this document,
and produces the expected output.}

This is an easy warm-up assignment,
we expect that you will be able to complete it
reasonably quickly.
% even if you do not have previous Python experience.
The programs you have to code are simple,
so if you find yourself coding a very complex program,
you have probably taken a wrong approach.

We will not grade the quality
or the speed of your code for this assignment,
however we expect that
(a)  it will run (no syntax or runtime errors are allowed);
and (b) it will produce
the \emph{exact} expected output in a \emph{reasonable} time:
less than $5$ minutes to create/load the full index
and less than $10$ seconds to answer each query.

If your code adheres to this specification,
you will get full credit.
We will penalize you if your code
produces an incorrect output
and/or it does not fully respect this specification.
The penalization will be proportional
to the seriousness of the problem.

\emph{You should submit this assignment
in order to be registered
in the evaluation system of the Course.}

\vspace{1cm}
\center{\large{Have fun!}}

\end{document}
